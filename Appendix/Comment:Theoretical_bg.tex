\subsection{Theoretical background}
 The alchemical free energy perturbation (AFEP) method can predict ligand-protein binding affinities efficiently thanks to fictitious intermediate states. A thermodynamic cycle is established through a series of nonphysical transformations between initial and final states. The free energy change {$\Delta G$} around the cycle is zero because the free energy is a state function. \cite{Kollman1993, Chodera2011, Deng2018, Bhati2018, Kuhn2020}.
 Nevertheless, the selection of intermediate states conditions the accuracy and efficiency of the free energy assessment. AFEP with double decoupling is a formally exact approach, but numerical convergence can be difficult to achieve\cite{Pohorille2010}.
 Although an abundance of comprehensive essays are available about the theory of affinity calculations, we briefly review the methodology background \cite{Bhandarkar2009, VanGunsteren2002, Chipot2006}.

 We describe an alchemical transformation from state $A$ to state $B$ through a different Hamiltonian for each state, $H_A$ and $H_B$, and more precisely, potential energy functions $U_A$ and $U_B$ (the kinetic energy does not change, and cancels out of the free energy calculations).

 The difference of free energy between two arbitrary states (say, a reference $r$ and a target $t$) can be calculated \textit{exactly} by the exponential formula:\cite{Zwanzig1954}
 \begin{align}\label{eq:exp_formula}
 \Delta F_{r\rightarrow t} &= -\frac{1}{\beta}\ln{ \left\langle \frac{e^{-\beta U_t(q, p)}} {e^{-\beta U_r(q, p)}}
 \right\rangle}_r \\
 &= -\frac{1}{\beta}\ln{ \left\langle e^{-\beta\Delta U_{r\rightarrow t}(q, p)} \right\rangle}_r
 \end{align}
 where $\beta$ is the inverse temperature $1/(k_B T)$, and the bracket indicates an average over $(q, p)$ in the canonical ensemble for the potential energy $U_r$, that is, an \textit{average in the reference state}.
 For any non-trivial perturbation, the exponential average in Eq.~(\ref{eq:exp_formula}) converges excruciatingly slowly, making this expression less than useful.
 In practice, to improve numerical convergence, the transformation from the reference to the target state is split into smaller steps, connecting a series of nonphysical intermediate states \cite{Beveridge1989}.
 Thus we can apply Eq.~(\ref{eq:exp_formula}) to many intermediate states between $A$ and $B$.

 The perturbation can be described continuously by a potential energy that depends on a “coupling parameter” $\lambda$ between 0 and 1.
 In the simplest case, it is a linear combination of the initial and final state potential energies:
 \begin{equation}\label{eq:U_lambda}
 U_\lambda(q, p)=(1-\lambda)U_A(q, p)+\lambda U_B(q, p)
 \end{equation}

 Calculating the difference of a free energy between the states through a series of intermediate transformations makes the exponential formula converge faster.
 There are also estimators termed \textit{Overlap Sampling} that improve convergence by introducing fictitious intermediate sates with better overlap with the states of interest and using them as target state in the exponential formula.
 These include simple overlap sampling (SOS) and Bennett's acceptance ratio (BAR). The SOS strategy simulates forward and a reverse sampling to an intermediate halfway between $\lambda_i$ and $\lambda_{i+1}$:\cite{Lu2004}
 \begin{equation}\label{eq:SOS}
 \Delta F_{i,i+1}=-\frac{1}{\beta}\ln{\frac
 {\left\langle e^{-\beta\Delta U_{i,i+1}/2} \right\rangle_{i}}
 {\left\langle e^{\beta\Delta U_{i,i+1}/2} \right\rangle_{i+1}}
 }
 \end{equation}
 BAR determines the optimal intermediate state by an iterative procedure, minimizing the statistical error\cite{Bennett1976,Lu2004,Lu2004a}.


 Thermodynamic integration is an alternative general-purpose method for the calculation of free energies along a continuous parameter, which may be a geometric coordinate or an alchemical coupling parameter. \cite{Mitchell1991,VanGunsteren2002,Jorge2010} 
 Thermodynamic integration can be written as:
 \begin{equation}\label{eq:TI}
 \Delta F=\int_0^1\left\langle\frac{\partial U_\lambda(q,p)}{\partial\lambda}\right\rangle_\lambda d\lambda
 \end{equation}

 In all alchemical transformations where particles are annihilated or decoupled from the environment, a singularity problem referred to as “end-point catastrophe” can occurred at the end points where repulsive nonbonded interactions disappear.
 With linear scaling of the repulsive potential, the free energy derivative with respect to $\lambda$ is infinite at the end-points, making the free energy estimation unreliable.
 To avoid this singularity, the interaction potential energy is transformed as a nonlinear function of $\lambda$, to behave as a so-called “soft-core potential” \cite{Beutler1994,Zacharias1994}.

 In the double decoupling method (DDM), the ligand is decoupled separately from the binding site and from bulk solution, by turning off the ligand's van der Waals and electrostatic interactions \cite{Straatsma1992, Boresch2003, Nishikawa2018}.
 Restraining potentials are usually applied to the ligand in the binding cavity to prevent it from diffusing away from the site and sampling a large volume of irrelevant configurations when it becomes mostly decoupled\cite{Deng2009}.
 It is necessary to account for the effect of the restraining potential when computing the standard free energy of binding.
 This contribution can be estimated numerically by a restraint free-energy perturbation (RFEP) simulation.\cite{Mobley2006, Deng2009, Salari2018, Sakae2020}.
